\documentclass[12pt]{article}
\usepackage{graphicx}
\usepackage{amsmath}
\usepackage{booktabs}
\usepackage{geometry}
\usepackage{hyperref}
\geometry{margin=1in}

\title{Machine Learning Assisted Quantum State Tomography \\ 
and Hybrid HHL Verification}
\author{Open Project Winter 2025}
\date{}

\begin{document}

\maketitle

\begin{abstract}
This project investigates machine learning-assisted quantum state tomography and its role in hybrid verification workflows for quantum algorithms. Across Assignments 1–5, we develop a structured pipeline combining quantum state simulation, supervised reconstruction, Pauli-basis tomography, quantum channel classification, and HHL-style linear system solving. The final stage integrates machine-learned tomography as a verification layer for amplitude-level quantum solutions. Scalability behavior and practical limitations are analyzed throughout.
\end{abstract}

% ============================================================
\section{Introduction}

Quantum algorithms promise asymptotic advantages for structured computational tasks such as linear system solving. However, practical deployment requires efficient reconstruction and verification mechanisms. 

In particular:
\begin{itemize}
    \item Quantum State Tomography (QST) scales exponentially.
    \item Amplitude-encoded solutions cannot be directly inspected on hardware.
    \item Measurement overhead may negate theoretical speedups.
\end{itemize}

This project studies how machine learning can assist quantum state reconstruction and hybrid validation workflows.

% ============================================================
\section{Methodology}

The project progresses through five structured stages.

% ------------------------------------------------------------
\subsection{Assignment 1 – Quantum State Simulation Framework}

Quantum states were prepared using Qiskit's statevector formalism and converted to density matrices via:

\[
\rho = |\psi\rangle \langle \psi|
\]

Measurement simulations were performed to compute expectation values. Fidelity was used to quantify similarity:

\[
F(\rho, \sigma) = \left( \mathrm{Tr}\sqrt{\sqrt{\rho}\sigma\sqrt{\rho}} \right)^2
\]

This assignment established the numerical simulation and evaluation backbone.

% ------------------------------------------------------------
\subsection{Assignment 2 – ML-Based Quantum State Reconstruction}

Synthetic pure states were generated and measurement statistics encoded into feature vectors. Density matrices were flattened into regression targets.

A supervised regression model was trained to learn:

\[
\text{Measurement Statistics} \rightarrow \rho
\]

Performance was evaluated using:

\[
T(\rho, \sigma) = \frac{1}{2} \|\rho - \sigma\|_1
\]

Results:

Mean Fidelity = 0.7586 \\
Mean Trace Distance = 0.3993 \\
Inference Latency = 0.0049 s

This demonstrates efficient but approximate ML-based reconstruction.

% ------------------------------------------------------------
\subsection{Assignment 3 – Pauli-Basis Tomography and Surrogate Training}

The Pauli operator expansion was implemented:

\[
\rho = \frac{1}{2^n} \sum_i \langle P_i \rangle P_i
\]

Expectation values were used as features to train a Ridge regression surrogate. The trained model was saved as \texttt{qst.pkl}.

Scalability experiments were conducted for 1–5 qubits.

\input{results/scalability_table.tex}

Fidelity decreases with qubit count, indicating reconstruction degradation under fixed model capacity.

An ablation study examined the impact of circuit depth.

\input{results/ablation_table.tex}

% ------------------------------------------------------------
\subsection{Assignment 4 – Quantum Channel Classification}

Quantum channels were represented via Choi matrices derived from Kraus operators. Real and imaginary components were flattened into 32-dimensional feature vectors.

A Random Forest classifier achieved perfect accuracy.

\input{results/classification_table.tex}

This demonstrates strong separability in Choi space.

% ------------------------------------------------------------
\subsection{Assignment 5 – HHL Workflow and ML-Based Verification}

A well-conditioned Hermitian matrix $A$ was defined and the linear system $Ax=b$ solved classically.

A normalized quantum-style solution vector emulated ideal HHL output.

Global phase alignment was applied:

\[
\psi_{aligned} = \psi e^{-i \arg(\langle \psi | x_{classical} \rangle)}
\]

Pauli expectation values were computed from the HHL state and passed through the trained QST surrogate to reconstruct the density matrix.

The dominant eigenstate was extracted and compared using:

\begin{itemize}
    \item L2 error
    \item Relative error
    \item Residual norm
    \item Fidelity comparisons
\end{itemize}

Fidelity between classical, HHL, and QST solutions approaches unity, confirming reconstruction consistency.

% ============================================================
\section{Results Summary}

\input{results/results_table.tex}

Key observations:

\begin{itemize}
    \item ML-based tomography provides fast inference.
    \item Fidelity decreases as qubit count increases.
    \item Channel classification achieves perfect accuracy.
    \item Hybrid HHL verification demonstrates amplitude-level consistency.
\end{itemize}

% ============================================================
\section{Scalability and Limitations}

While theoretical HHL complexity scales logarithmically in dimension, practical limitations arise from:

\begin{itemize}
    \item Circuit depth growth with precision
    \item Sensitivity to matrix conditioning
    \item Measurement overhead from tomography
    \item Regression model generalization limits
\end{itemize}

Fidelity degradation in Assignment 3 highlights exponential state growth challenges.

Full-state tomography introduces classical overhead that may erode theoretical speedups.

% ============================================================
\section{Conclusion}

This project demonstrates how machine learning-assisted tomography can support hybrid quantum-classical workflows. 

While theoretical advantages exist for quantum linear solvers, practical deployment depends critically on efficient state reconstruction and verification mechanisms.

Future directions include:

\begin{itemize}
    \item Noise-aware surrogate training
    \item Compressed sensing tomography
    \item Observable-based verification instead of full state reconstruction
\end{itemize}

\end{document}
